\documentclass{article}
\usepackage[english]{babel}
\usepackage[a4paper,top=2.54cm,bottom=2.54cm,left=2.54cm,right=2.54cm,marginparwidth=1.75cm]{geometry}
\usepackage{amsmath}
\usepackage{graphicx}
\usepackage{amsfonts}
\usepackage{amssymb}
\usepackage{enumerate}
\usepackage{enumitem}
\usepackage[colorlinks=true, allcolors=blue]{hyperref}
\title{Linear Algebra: Homework 8}
\begin{document}
\maketitle
\subsection*{Question 1.}
Without writing $A$, find an eigenvalue of $A$ and describe the eigenspace:
\begin{enumerate} [label=(\arabic*)]
    \item $T$ is the transformation on $\mathbf{R}^2$ that reflects points across some line through the origin.
    \item $T$ is the transformation on $\mathbf{R}^3$ that rotates points about some line through the origin. 
\end{enumerate}
\subsection*{Solution 1.}
\begin{enumerate} [label=(\arabic*)]
    \item $A$ would have eigenvalues of $-1$ and $1$, where the eigenspace correspond to $\lambda=-1$ are all vectors with direction perpendicular to the line, and the eigenspace correspond to $\lambda=1$ are all vectors parallel to the line.
    \item $A$ would have eigenvalues of $1,e^{i\theta}$ and $e^{-i\theta}$, where $\theta$ is the angle of rotation. The eigenspace correspond to $\lambda=1$ are all vectors parallel to the line axis of rotation. The eigenspaces correspond to the two other eigenvalues are not discussed, since these eigenvectors do not belong to $\mathbf{R}^3$.
\end{enumerate}
\subsection*{Question 2.}
Let $\mathbf{u}$ and $\mathbf{v}$ be eigenvectors of a matrix $A$, with eigenvalues $\lambda$ and $\mu$ respectively. Let $c_1$, $c_2$ be scalars. Define $\mathbf{x}_k=c_1\lambda^k\mathbf{u}+c_2\mu^k\mathbf{v},k=0,1,2,\cdots.$
\begin{enumerate} [label=(\arabic*)]
    \item What is $\mathbf{x}_{k+1}$, by definition?
    \item Compute $A\mathbf{x}_k$ from the formula for $\mathbf{x}_k$, and show that $A\mathbf{x}_k=\mathbf{x}_{k+1}$.
\end{enumerate}
\subsection*{Solution 2.}
\begin{enumerate} [label=(\arabic*)]
    \item $\mathbf{x}_{k+1}=c_1\lambda^{k+1}\mathbf{u}+c_2\mu^{k+1}\mathbf{v}$
    \item $A\mathbf{x}_k=A(c_1\lambda^k\mathbf{u}+c_2\mu^k\mathbf{v})=c_1\lambda^kA\mathbf{u}+c_2\mu^kA\mathbf{v}=c_1\lambda^k(\lambda\mathbf{u})+c_2\mu^k(\mu\mathbf{v})=c_1\lambda^{k+1}\mathbf{u}+c_2\mu^{k+1}\mathbf{v}=\mathbf{x}_{k+1}$
\end{enumerate}
\subsection*{Question 3.}
Find the characteristic polynomial and eigenvalues of the matrices
\[\left[\begin{array}{rr}
5 & 3 \\
-4 & 4
\end{array}\right],\left[\begin{array}{rr}
7 & -2 \\
2 & 3
\end{array}\right].\]
\subsection*{Solution 3.}
Let $A=\left[\begin{array}{rr}
5 & 3 \\
-4 & 4
\end{array}\right],B=\left[\begin{array}{rr}
7 & -2 \\
2 & 3
\end{array}\right]$. Then, the characteristic polynomials are
\[p_A(\lambda)=\left\vert\begin{array}{rr}
5-\lambda & 3 \\
-4 & 4-\lambda
\end{array}\right\vert=20-9\lambda+\lambda^2+12=\lambda^2-9\lambda+32\]
and
\[p_B(\lambda)=\left\vert\begin{array}{rr}
7-\lambda & -2 \\
2 & 3-\lambda
\end{array}\right\vert=21-10\lambda+\lambda^2+4=\lambda^2-10\lambda+25\]
respectively.\newline
Eigenvalues $\lambda_1,\lambda_2$ of $A$ satisfies $p_A(\lambda_1)=p_A(\lambda_2)=0$, in which 
\[\lambda_1=\frac{9-\sqrt{81-128}}{2}=4.5-\frac{1}{2}\sqrt{47}i,\lambda_2=4.5+\frac{1}{2}\sqrt{15}i\]
Eigenvalue $\lambda_1$ satisfies $p_B(\lambda_1)=(\lambda_1-5)^2=0\Rightarrow \lambda_1=5$, with algebric multiplicity of 2.
\subsection*{Question 4.}
Find the characteristic polynomial of the following matrices, with the cofactor expansion of the determinant:
\[\left[\begin{array}{rrr}
6 & -2 & 0 \\
-2 & 9 & 0 \\
5 & 8 & 3
\end{array}\right],\left[\begin{array}{rrr}
5 & -2 & 3 \\
0 & 1 & 0 \\
6 & 7 & -2
\end{array}\right].\]
\subsection*{Solution 4.}
Let $A=\left[\begin{array}{rrr}
6 & -2 & 0 \\
-2 & 9 & 0 \\
5 & 8 & 3
\end{array}\right],B=\left[\begin{array}{rrr}
5 & -2 & 3 \\
0 & 1 & 0 \\
6 & 7 & -2
\end{array}\right]$. Then, the characteristic polynomials are
\[p_A(\lambda)=\left\vert\begin{array}{rrr}
6-\lambda & -2 & 0 \\
-2 & 9-\lambda & 0 \\
5 & 8 & 3-\lambda
\end{array}\right\vert=(3-\lambda)\left\vert\begin{array}{rr}
6-\lambda & -2 \\
-2 & 9-\lambda
\end{array}\right\vert=(3-\lambda)(50-15\lambda+\lambda^2)=-\lambda^3+18\lambda^2-95\lambda+150\]
\[p_B(\lambda)=\left\vert\begin{array}{rrr}
5-\lambda & -2 & 3 \\
0 & 1-\lambda & 0 \\
6 & 7 & -2-\lambda
\end{array}\right\vert=(1-\lambda)\left\vert\begin{array}{rr}
5-\lambda & 3 \\
6 & -2-\lambda
\end{array}\right\vert=(1-\lambda)(-28-3\lambda+\lambda^2)=-\lambda^3+4\lambda^2+25\lambda-28\]
\subsection*{Question 5.}
Diagonalize the folowing matrices if possible, it is known that the first matrix has eigenvalues $\lambda=2,8$, and that the second matrix has eigenvalues $\lambda=2,1$:
\[\left[\begin{array}{rrr}
4 & 2 & 2 \\
2 & 4 & 2 \\
2 & 2 & 4
\end{array}\right],\left[\begin{array}{rrr}
0 & -4 & -6 \\
-1 & 0 & -3 \\
1 & 2 & 5
\end{array}\right].\]
\subsection*{Solution 5.}
Let $A=\left[\begin{array}{rrr}
4 & 2 & 2 \\
2 & 4 & 2 \\
2 & 2 & 4
\end{array}\right]$, $B=\left[\begin{array}{rrr}
0 & -4 & -6 \\
-1 & 0 & -3 \\
1 & 2 & 5
\end{array}\right]$.\newline
For $A$, \newline
eigenvectors $\Vec{v_1}$ correspond to $\lambda_1=2$ satisfies
\[(A-2I)\Vec{v_1}=0\]
Solve the above equation by augmented matrix, we have
\[\left[\begin{array}{rrrr}
2 & 2 & 2 & 0 \\
2 & 2 & 2 & 0 \\
2 & 2 & 2 & 0
\end{array}\right]\sim \left[\begin{array}{rrrr}
1 & 1 & 1 & 0 \\
0 & 0 & 0 & 0 \\
0 & 0 & 0 & 0
\end{array}\right]\]
Thus $\Vec{v_1}=t_1\left[\begin{array}{r}-1\\1\\0\end{array}\right]+t_2\left[\begin{array}{r}-1\\0\\1\end{array}\right]$, where $t_1,t_2\in \mathbf{R}$ are parameters.\newline
eigenvectors $\Vec{v_2}$ correspond to $\lambda_2=8$ satisfies
\[(A-8I)\Vec{v_2}=0\]
Solve the above equation by augmented matrix, we have
\[\left[\begin{array}{rrrr}
-4 & 2 & 2 & 0 \\
2 & -4 & 2 & 0 \\
2 & 2 & -4 & 0
\end{array}\right]\sim\left[\begin{array}{rrrr}
1 & 1 & -2 & 0 \\
1 & -2 & 1 & 0 \\
-2 & 1 & 1 & 0
\end{array}\right]\sim \left[\begin{array}{rrrr}
1 & 1 & -2 & 0 \\
0 & -3 & 3 & 0 \\
0 & 3 & -3 & 0
\end{array}\right]\sim\left[\begin{array}{rrrr}
1 & 1 & -2 & 0 \\
0 & 1 & -1 & 0 \\
0 & 0 & 0 & 0
\end{array}\right]
\]
Thus, $\Vec{v_2}=t_3\left[\begin{array}{r} 1 \\ 1 \\ 1 \end{array}\right]$ where $t_3\in\mathbf{R}$ is a parameter. \newline
Let $A=PDP^{-1}$, where $D=\left[\begin{array}{rrr}
2\\ & 2 \\ && 8 
\end{array}\right]$.
Then, one possible way to construct $P$ is $P=\left[\begin{array}{rrr}
1 & 1 & 1 \\
-1 & 0 &1 \\ 0 & -1 & 1 \end{array}\right]$.
\[[P|I_3]\sim 
\left[\begin{array}{rrrrrr}
1 & 1 & 1 & 1 \\
-1 & 0 & 1 &  & 1\\
0 & -1 & 1 &  &  & 1
\end{array}\right]\sim
\left[\begin{array}{rrrrrr}
1 & 1 & 1 & 1 \\
0 & 1 & 2 & 1 & 1\\
0 & -1 & 1 &  &  & 1
\end{array}\right]\sim
\left[\begin{array}{rrrrrr}
1 & 1 & 1 & 1 \\
0 & 1 & 2 & 1 & 1\\
0 & 0 & 3 & 1 & 1 & 1
\end{array}\right]\sim
\]\[
\left[\begin{array}{rrrrrr}
1 & 1 & 1 & 1 \\
0 & 1 & 2 & 1 & 1\\
0 & 0 & 1 & 1/3 & 1/3 & 1/3
\end{array}\right]\sim
\left[\begin{array}{rrrrrr}
1 & 1 & 0 & 2/3 & -1/3 & -1/3 \\
0 & 1 & 0 & 1/3 & 1/3 & -2/3\\
0 & 0 & 1 & 1/3 & 1/3 & 1/3
\end{array}\right]\sim
\left[\begin{array}{rrrrrr}
1 & 0 & 0 & 1/3 & -2/3 & -1 \\
0 & 1 & 0 & 1/3 & 1/3 & -2/3\\
0 & 0 & 1 & 1/3 & 1/3 & 1/3
\end{array}\right]
\]
Hence,
\[A=\left[\begin{array}{rrr}
1 & 1 & 1 \\
-1 & 0 &1 \\ 0 & -1 & 1 \end{array}\right]\left[\begin{array}{rrr}
2\\ & 2 \\ && 8 
\end{array}\right]\left[\begin{array}{rrr}
 1/3 & -2/3 & -1 \\
 1/3 & 1/3 & -2/3\\
 1/3 & 1/3 & 1/3
\end{array}\right]\]
For $B$,\newline
eigenvectors $\Vec{v_1}$ correspond to $\lambda_1=2$ satisfies
\[(B-2I)\Vec{v_1}=0\]
Solve the above equation by augmented matrix, we have
\[
\left[\begin{array}{rrrr}
-2 & -4 & -6 & 0 \\
-1 & -2 & -3 & 0  \\
 1 &  2 &  3 & 0 
\end{array}\right]
\sim
\left[\begin{array}{rrrr}
 1 &  2 & 3 & 0  \\
-1 & -2 & -3 & 0  \\
-1 & -2 & -3 & 0 
\end{array}\right]
\sim 
\left[\begin{array}{rrrr}
 1 &  2 & 3 & 0  \\
 0 &  0 & 0 & 0  \\
 0 &  0 & 0 & 0 
\end{array}\right]
\]
Thus, $\Vec{v_1}=t_1\left[\begin{array}{r}-2\\1\\0\end{array}\right]+t_2\left[\begin{array}{r}-3\\0\\1\end{array}\right]$, where $t_1,t_2\in \mathbf{R}$ are parameters.\newline
eigenvectors $\Vec{v_1}$ correspond to $\lambda_1=1$ satisfies
\[(B-I)\Vec{v_1}=0\]
Solve the above equation by augmented matrix, we have
\[
\left[\begin{array}{rrrr}
-1 & -4 & -6 & 0 \\
-1 & -1 & -3 & 0  \\
 1 &  2 &  4 & 0 
\end{array}\right]
\sim
\left[\begin{array}{rrrr}
-1 & -4 & -6 & 0 \\
 0 &  3 &  3 & 0  \\
 0 & -2 & -2 & 0 
\end{array}\right]
\sim
\left[\begin{array}{rrrr}
-1 & -4 & -6 & 0 \\
 0 &  1 &  1 & 0  \\
 0 &  0 &  0 & 0 
\end{array}\right]
\]
Thus $\Vec{v_2}=t_3\left[\begin{array}{r}-2\\-1\\1\end{array}\right]$, where $t_3\in \mathbf{R}$ is a parameter.\newline
Let $A=PDP^{-1}$, where $D=\left[\begin{array}{rrr}
2\\ & 2 \\ && 1 
\end{array}\right]$.
Then, one possible way to construct $P$ is $P=\left[\begin{array}{rrr}
2 & 3 & 2 \\
-1 & 0 &1 \\ 
0 & -1 & -1 \end{array}\right]$.
\[[P|I_3]\sim 
\left[\begin{array}{rrrrrr}
2 & 3 & 2 & 1  \\
-1 & 0 & 1 &  & 1 \\
0 & -1 & -1 &  &  & 1
\end{array}\right]\sim
\left[\begin{array}{rrrrrr}
2 & 3 & 2 & 1  \\
0 & 3/2 & 2 & 1/2 & 1 \\
0 & -1 & -1 &  &  & 1
\end{array}\right]
\sim
\left[\begin{array}{rrrrrr}
2 & 3 & 2 & 1  \\
0 & 3/2 & 2 & 1/2 & 1 \\
0 & 0 & 1/3 & 1/3 & 2/3 & 1
\end{array}\right]\]
\[\sim
\left[\begin{array}{rrrrrr}
2 & 3 & 0 & -1 & -4 & -6  \\
0 & 3/2 & 0 & -3/2 & -3 & -6 \\
0 & 0 & 1 & 1 & 2 & 3
\end{array}\right]
\sim
\left[\begin{array}{rrrrrr}
2 & 0 & 0 & 2 & 2 & 6  \\
0 & 1 & 0 & -1 & -2 & -4 \\
0 & 0 & 1 & 1 & 2 & 3
\end{array}\right]
\sim
\left[\begin{array}{rrrrrr}
1 & 0 & 0 & 1 & 1 & 3  \\
0 & 1 & 0 & -1 & -2 & -4 \\
0 & 0 & 1 & 1 & 2 & 3
\end{array}\right]
\]
Hence, 
\[B=\left[\begin{array}{rrr}
2 & 3 & 2 \\
-1 & 0 &1 \\ 
0 & -1 & -1 \end{array}\right]\left[\begin{array}{rrr}
2\\ & 2 \\ && 1 
\end{array}\right]\left[\begin{array}{rrr}
1 & 1 & 3  \\
-1 & -2 & -4 \\
1 & 2 & 3
\end{array}\right]\]
\subsection*{Question 6.}
\begin{enumerate} [label=(\arabic*)]
    \item $A$ is a $5\times 5$ matrix with two eigenvalues, one eigenspace is 3-dimensional, and the other eigenspace is 2-dimensional. Is  $A$ diagonalizable? Why?
    \item $A$ is a $4\times 4$ matrix with three eigenvalues. One eigenspace is 1-dimensional, and one of the other eigenspaces is 2-dimensional. Is it possible that $A$ is not diagonalizable? Why?
\end{enumerate}
\subsection*{Solution 6.}
\begin{enumerate} [label=(\arabic*)]
    \item Yes. \newline
    The algebraic multiplicities of eigenvalues must not less than the dimension of the eigenspace correspond to that eigenvalue. Thus the first eigenvalue must have algebraic multiplicity not less than 3, and the latter not less than 2. \newline
    Moreover, the sum of algebraic multiplicities of the roots of a polynomial is the degree of it. The characteristic polynomial of $A$ is degree 5, hence we can conclude that the algebraic multiplicity of the first eigenvalue is 3 and the latter is 2. \newline
    Thus, for all eigenvalues of $A$, the dimensions of eigenspaces correspond to the eigenvalues is equal to the algebraic multiplicities of the eigenvalues in the characteristic polynomial.
    \item No\newline
    (The solution below assumes the three eigenvalues to be distinct.)\newline
    By using the conclusion in part 1, if the sum of the dimensions of eigenspaces is the dimension of the size of the matrix, i.e. 4 for $A$, then $A$ is diagonalizable.\newline
    The remaining eigenspace must have dimension 1,\newline
    since eigenspace, directly from the definition of eigenvectors, must have its dimension not less than 1,\newline
    nor more than the algebric multiplicity of the eigenvalue that the eigenspace corresponds to.\newline
    For this $4\times 4$ matrix, it has 3 eigenspaces, two of which are one-dimensional and correspond to the two eigenvalues with algebraic multiplicty one for each in the characteristic polynomial of $A$.\newline
    The other eigenspace is two-dimensional, eigenvalue corresponds to it has algebraic multiplicity of 2 in the characteristic polynomial.
\end{enumerate}
\subsection*{Question 7.}
Define $T:\mathbb{P}_2\to \mathbf{R}^3$ by $T(\mathbf{p})=\left[\begin{array}{r}
\mathbf{p}(-1) \\
\mathbf{p}(0) \\
\mathbf{p}(1) 
\end{array}\right]$.
\begin{enumerate} [label=(\arabic*)]
    \item Find the image under $T$ of $\mathbf{p}(t)=5+3t$.
    \item Show that $T$ is a linear transformation.
    \item Find the matrix for $T$ relative to the basis $\{1,t,t^2\}$ for $\mathbb{P}_2$ and the standard basis for $\mathbf{R}^3$.
\end{enumerate}
\subsection*{Solution 7.}
\begin{enumerate}
    \item $T(\mathbf{p})=\left[\begin{array}{r}2 \\5 \\8 \end{array}\right]$
    \item \[T(\sum_{i=1}^n c_i\mathbf{p}_i)=\left[\begin{array}{r} (\sum_{i=1}^n c_i\mathbf{p}_i)(-1)\\(\sum_{i=1}^n c_i\mathbf{p}_i)(0) \\(\sum_{i=1}^n c_i\mathbf{p}_i)(1) \end{array}\right]=\sum_{i=1}^n c_i\left[\begin{array}{r} \mathbf{p}_i(-1)\\\mathbf{p}_i(0) \\\mathbf{p}_i(1) \end{array}\right]=\sum_{i=1}^n c_i T(\mathbf{p_i})\]
    \item Let $A$ be the matrix for transformation $T$.
    \[A=[\begin{array}{rrr}T(1)&T(t)&T(t^2)\end{array}]=\left[\begin{array}{rrr}
    1 & -1 & 1 \\
    1 &  0 & 0 \\
    1 &  1 & 1 
    \end{array}\right]\]
\end{enumerate}
\subsection*{Question 8.}
Let $A$ and $B$ be similar square matrices, show that they have the same rank.
\subsection*{Solution 8.}
$A\sim B\Rightarrow (\exists P)(AP=PB)$.
Let $\mathbf{w}\in$ Ran$(B)$, and $\mathbf{v}$ is the preimage of $\mathbf{w}$. Then we have
\[AP\mathbf{v}=PB\mathbf{v}=P\mathbf{w}\]
Hence $P:$Ran($B)\to$ Ran($A$). \newline
As the linear transformation represented by $P$ is injective,\newline
rank($B$)=dim Ran($B)\leq$ dim Ran ($A)=$ rank($A$). (1) \newline
As matrix similarity is an equivalence relation, we also have rank($A)\leq$ rank($B$). (2) \newline
With (1) and (2), rank($A$)=rank($B$).$\blacksquare$
\subsection*{Question 9.}
\begin{enumerate} [label=(\arabic*)]
    \item The \textit{trace} of a square matrix $A$ is the sum of the diagonal entries of $A$, and is denoted Tr($A$). It can be verified that Tr($AB$)=Tr($BA$) for any two $n\times n$ matrices $A$ and $B$. Show that if $A$ and $B$ are similar, then Tr($A$)=Tr($B$).
    \item Suppose that $A$ is diagonalizable, show that Tr($A$) equals the sum of the eigenvalues of $A$.
\end{enumerate}
\subsection*{Solution 9.}
\begin{enumerate} [label=(\arabic*)]
    \item $A\sim B \Rightarrow (\exists P)(A=P^{-1}BP)\Rightarrow$ Tr($A$)=Tr($P^{-1}(BP)$)=Tr($(BP)P^{-1}$)=Tr($B$).
    \item $A$ is diagonalizable, so $A=P\Lambda P^{-1}$, where $P$ is a matrix with columns of eigenvectors of $A$ and $\Lambda$ is a matrix with eigenvalues of $A$ on the main diagonal and zero elsewhere. Hence, 
    \[Tr(A)=Tr((P\Lambda)P^{-1})=Tr(P^{-1}(P\Lambda))=Tr(\Lambda)=\sum _{i=1}^n (\Lambda)_{ii}=\sum_{i=1}^n \lambda_i\]
\end{enumerate}
\subsection*{Question 10.}
\begin{enumerate} [label=(\arabic*)]
    \item Diagonalize the matrix $\left[\begin{array}{rr}
    5 & -2 \\
    1 & 3
    \end{array}\right]$ in the realm of complex matrices if possible.
    \item Express the matrix $\left[\begin{array}{rr}
    1 & 5 \\
    -2 & 3
    \end{array}\right]$ in the form $PCP^{-1}$, with $P$ an invertible real matrix and $C$ of the form $\left[\begin{array}{rr}
    a & -b \\
    b & a
    \end{array}\right]$ with a,b \textit{real}.
\end{enumerate}
\subsection*{Solution 10.}
\begin{enumerate} [label=(\arabic*)]
    \item 
    \[\left\vert\begin{array}{rr}
    5-\lambda & -2 \\
    1 & 3-\lambda 
    \end{array}\right\vert=17-8\lambda+\lambda^2=0\Rightarrow \lambda_{1,2}=4\pm i\]
    Finding eigenvector corresponds to $\lambda_1=4-i$ leads to the calculation of augmented matrix 
    \[\left[\begin{array}{rrr}
    1+i & -2 & 0 \\
    1 & -1+i & 0 
    \end{array}\right]\sim \left[\begin{array}{rrr}
    1 & -1+i & 0 \\
    1+i & -2 & 0 
    \end{array}\right]\sim \left[\begin{array}{rrr}
    1 & -1+i & 0 \\
    0 & 0 & 0 
    \end{array}\right]\]
    Hence that eigenvector will be $t_1\left[\begin{array}{r}
    1-i  \\ 1 \end{array}\right]$, where $t_1\in\mathbf{C}$ is a parameter.
    Finding eigenvector corresponds to $\lambda_1=4+i$ leads to the calculation of augmented matrix 
    \[\left[\begin{array}{rrr}
    1-i & -2 & 0 \\
    1 & -1-i & 0 
    \end{array}\right]\sim \left[\begin{array}{rrr}
    1 & -1-i & 0 \\
    1-i & -2 & 0 
    \end{array}\right]\sim \left[\begin{array}{rrr}
    1 & -1-i & 0 \\
    0 & 0 & 0 
    \end{array}\right]\]
    Hence that eigenvector will be $t_2\left[\begin{array}{r}
    1+i  \\ 1 \end{array}\right]$, where $t_2\in\mathbf{C}$ is a parameter.\newline
    Let $\left[\begin{array}{rr}
    5 & -2 \\
    1 & 3
    \end{array}\right]=PDP^{-1}$, where $D=\left[\begin{array}{rr}
    4-i \\
     & 4+i
    \end{array}\right].$ One possible way to construct $P$ is $P=\left[\begin{array}{rr}
    1-i & 1+i \\
    1 & 1
    \end{array}\right]$.
    \[[P|I_2]\sim \left[\begin{array}{rrrr}
    1-i & 1+i & 1 \\
    1 & 1 && 1
    \end{array}\right]\sim
    \left[\begin{array}{rrrr}
    1-i & 1+i & 1 & 0\\
    0 & -1 & -1+i & 1
    \end{array}\right]
    \sim
    \left[\begin{array}{rrrr}
    1 & i & (1+i)/2 & 0 \\
    0 & 1 & 1-i & -1
    \end{array}\right]
    \]
    \[
    \sim
    \left[\begin{array}{rrrr}
    1 & 0 & -(1+i)/2 & i \\
    0 & 1 & 1-i & -1
    \end{array}\right]
    \]
    Hence, \[
    \left[\begin{array}{rr}
    5 & -2 \\
    1 & 3
    \end{array}\right]=
    \left[\begin{array}{rr}
    1-i & 1+i \\
    1 & 1
    \end{array}\right]
    \left[\begin{array}{rr}
    4-i \\
     & 4+i
    \end{array}\right]
    \left[\begin{array}{rr}
     -(1+i)/2 & i \\
     1-i & -1
    \end{array}\right]
    \]
    %----------------------------
    \item 
    % \begin{align}
    % \left[\begin{array}{rr}
    % 5 & -2 \\
    % 1 & 3
    % \end{array}\right]
    % \left[\begin{array}{r}1+i  \\ 1 \end{array}\right]=
    % \left[\begin{array}{rr}
    % 5 & -2 \\
    % 1 & 3
    % \end{array}\right]\left(\left[\begin{array}{r}1 \\ 1 \end{array}\right]+i\cdot \left[\begin{array}{r}1\\0\end{array}\right]\right)=
    % \left[\begin{array}{rr}
    % 5 & -2 \\
    % 1 & 3
    % \end{array}\right]\left[\begin{array}{r}1 \\ 1 \end{array}\right]+i\cdot \left[\begin{array}{rr}
    % 5 & -2 \\
    % 1 & 3
    % \end{array}\right]\left[\begin{array}{r}1 \\0\end{array}\right] \\
    % (4+i)\left[\begin{array}{r}1+i\\ 1 \end{array}\right]=\left[\begin{array}{r}3+5i\\4+i \end{array}\right]=\left[\begin{array}{r}3\\4 \end{array}\right]+i\cdot \left[\begin{array}{r}5\\1 \end{array}\right]
    % \end{align}
    For any $A\in\mathbf{R}^{2\times 2}$, Let $\lambda\in\mathbf{C}$ be one of the eigenvalue of $A$.By definition,
    \[A\mathbf{v}=\lambda\mathbf{v}\Rightarrow A(\Re(\mathbf{v})+i\cdot \Im(\mathbf{v}))=A\Re(\mathbf{v})+i\cdot A\Im(\mathbf{v})\]
    \[=(\Re(\lambda)+i\cdot\Im(\lambda))(\Re(\mathbf{v})+i\cdot\Im(\mathbf{v}))=(\Re(\lambda)\Re(\mathbf{v})-\Im(\lambda)\Im(\mathbf{v}))+i\cdot(\Im(\lambda)\Re(\mathbf{v})+\Re(\lambda)\Im(\mathbf{v})),\]
    where $\mathbf{v}$ is the eigenvector correspond to $\lambda$.
    \[\Rightarrow A[\begin{array}{rr}\Re(\mathbf{v})&\Im(\mathbf{v})\end{array}]=[\begin{array}{rr}\Re(\lambda)\Re(\mathbf{v})-\Im(\lambda)\Im(\mathbf{v})&\Im(\lambda)\Re(\mathbf{v})+\Re(\lambda)\Im(\mathbf{v})\end{array}]\]
    \[=[\begin{array}{rr}\Re(\mathbf{v})&\Im(\mathbf{v})\end{array}]\left[\begin{array}{rr}
    \Re(\lambda) & \Im(\lambda) \\
    -\Im(\lambda) & \Re(\lambda)
    \end{array}\right]\]
    Consider the given matrix, we have
    \[\left\vert\begin{array}{rr}
     1-\lambda & 5 \\
     -2 & 3-\lambda
     \end{array}\right\vert=13-4\lambda+\lambda^2=0\Rightarrow \lambda=2\pm 3i\]
     Choose $\lambda=2-3i$, then
     \[\left[\begin{array}{rr}
    -1+3i  & 5 \\
    -2  & 1+3i
     \end{array}\right]\Vec{v}=0\Rightarrow \Vec{v}=\left[\begin{array}{r}
     1+3i  \\2\end{array}\right]\]
     Hence,
     \[\left[\begin{array}{rr}
     5 & -2 \\
     1 & 3
     \end{array}\right]=\left[\begin{array}{rr}
     1 & 3 \\
     2 & 0
     \end{array}\right]
     \left[\begin{array}{rr}
     2 & -3 \\
     3 & 2
     \end{array}\right]
     \left[\begin{array}{rr}
     1 & 3 \\
     2 & 0
     \end{array}\right]^{-1}
     \]
     \[
     =
     \left[\begin{array}{rr}
     1 & 3 \\
     2 & 0
     \end{array}\right]
     \left[\begin{array}{rr}
     2 & -3 \\
     3 & 2
     \end{array}\right]
     \left[\begin{array}{rr}
     0 & 1/2 \\
     1/3 & -1/6
     \end{array}\right]
     \]
\end{enumerate}
\end{document} 