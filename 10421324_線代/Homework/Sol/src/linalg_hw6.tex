\documentclass{article}
\usepackage[english]{babel}
\usepackage[a4paper,top=2.54cm,bottom=2.54cm,left=2.54cm,right=2.54cm,marginparwidth=1.75cm]{geometry}
\usepackage{amsmath}
\usepackage{graphicx}
\usepackage{amsfonts}
\usepackage{amssymb}
\usepackage{enumerate}
\usepackage{enumitem}
\usepackage[colorlinks=true, allcolors=blue]{hyperref}

\title{Linear Algebra: Homework 6}
\begin{document}
\maketitle
\subsection*{Question 1.}
In $\mathbb{P}_2$, find the transition matrix from the basis $\mathcal{B}=\{1-3t^2,2+t-5t^2,1+2t\}$ to the standard basis. Then write $t^2$ as a linear combination of the polynomials in $\mathcal{B}$.
\subsection*{Solution 1.}
The transition matrix is actually the "column vectors" of the polynomials, as $\mathcal{B}$ is already shown in standard basis, which is
\[\left[\begin{array}{rrr}
1 & 2 & 1 \\
0 & 1 & 2  \\
-3 & -5 & 0 
\end{array}\right]\]
Express $t^2$ in $\mathcal{B}$ requires the inverse of the transition matrix.
\[\left[\begin{array}{rrrrrr}
1 & 2 & 1 & 1 &0 &0\\
0 & 1 & 2 & 0 & 1 & 0 \\
-3 & -5 & 0 & 0 & 0 & 1
\end{array}\right]
\sim
\left[\begin{array}{rrrrrr}
1 & 2 & 1 & 1 &0 &0\\
0 & 1 & 2 & 0 & 1 & 0 \\
0 & 1 & 3 & 3 & 0 & 1
\end{array}\right]
\sim
\left[\begin{array}{rrrrrr}
1 & 2 & 1 & 1 &0 &0\\
0 & 1 & 2 & 0 & 1 & 0 \\
0 & 0 & 1 & 3 & -1 & 1
\end{array}\right]
\]
\[
\sim
\left[\begin{array}{rrrrrr}
1 & 2 & 0 & -2 & 1 &-1\\
0 & 1 & 0 & -6 & 3 & -2 \\
0 & 0 & 1 & 3 & -1 & 1
\end{array}\right]
\sim
\left[\begin{array}{rrrrrr}
1 & 0 & 0 & 10 & -5 &3\\
0 & 1 & 0 & -6 & 3 & -2 \\
0 & 0 & 1 & 3 & -1 & 1
\end{array}\right]
\]
\[\left[\begin{array}{rrr}
10 & -5 &3\\
-6 & 3 & -2 \\
 3 & -1 & 1
\end{array}\right]\left[\begin{array}{r}0 \\ 0 \\ 1 \end{array}\right]=\left[\begin{array}{r}3 \\ -2 \\ 1 \end{array}\right]\]
Therefore, 
\[t^2=3(1-3t^2)-2(2+t-5t^2)+(1+2t)\]
\subsection*{Question 2.}
Let $P=\left[\begin{array}{rrr}
1 & 2 &-1 \\
-3 & -5 & 0 \\
4 & 6 & 1
\end{array}\right]$, and
\[\Vec{v_1}=\left[\begin{array}{c}-2\\2\\3\end{array}\right],\Vec{v_2}=\left[\begin{array}{c}-8\\5\\2\end{array}\right],\Vec{v_3}=\left[\begin{array}{c}-7\\2\\6\end{array}\right].\]
\begin{enumerate} [label={(\arabic*)}]
\item Find a basis $\{\Vec{u_1},\Vec{u_2},\Vec{u_3}\}$ in $\mathbf{R}^3$ such that $P$ is the transition matrix from $\{\Vec{u_1},\Vec{u_2},\Vec{u_3}\}$ to the basis $\{\Vec{v_1},\Vec{v_2},\Vec{v_3}\}$.
\item Find a basis $\{\Vec{w_1},\Vec{w_2},\Vec{w_3}\}$ in $\mathbf{R}^3$ such that $P$ is the transition matrix from $\{\Vec{v_1},\Vec{v_2},\Vec{v_3}\}$ to the basis $\{\Vec{w_1},\Vec{w_2},\Vec{w_3}\}$.
\end{enumerate}
\subsection*{Solution 2.}
Calculate $P^{-1}$ first.
\[\left[\begin{array}{rrrrrr}
1 & 2 &-1&1&0&0 \\
-3 & -5 & 0&0&1&0 \\
4 & 6 & 1&0&0&1
\end{array}\right]
\sim 
\left[\begin{array}{rrrrrr}
1 & 2 &-1&1&0&0 \\
0 & 1 & -3&3&1&0 \\
0 & -2 & 5&-4&0&1
\end{array}\right]
\sim 
\left[\begin{array}{rrrrrr}
1 & 2 &-1&1&0&0 \\
0 & 1 & -3&3&1&0 \\
0 & 0 & -1&2&2&1
\end{array}\right]
\]
\[
\sim 
\left[\begin{array}{rrrrrr}
1 & 2 &-1&1&0&0 \\
0 & 1 & -3&3&1&0 \\
0 & 0 & 1&-2&-2&-1
\end{array}\right]
\sim 
\left[\begin{array}{rrrrrr}
1 & 2 &0&-1&-2&-1 \\
0 & 1 & 0&-3&-5&-3 \\
0 & 0 & 1&-2&-2&-1
\end{array}\right]
\sim 
\left[\begin{array}{rrrrrr}
1 & 0 &0&5&8&5 \\
0 & 1 & 0&-3&-5&-3 \\
0 & 0 & 1&-2&-2&-1
\end{array}\right]
\]
\begin{enumerate} [label={(\arabic*)}]
\item
    Then entries of each column in $P$ are the coefficients of the linear combination of $\Vec{v_i}$ $(i=1,2,3)$ of vectors $\Vec{u_i}$ $(i=1,2,3)$. So, 
     \[\left[\begin{array}{rrr}\Vec{u_1}&\Vec{u_2}&\Vec{u_3}\end{array}\right]=\left[\begin{array}{rrr}\Vec{v_1}&\Vec{v_2}&\Vec{v_3}\end{array}\right]P\]
    \[=\left[\begin{array}{rrr}-2&-8&-7\\2&5&2\\3&2&6\end{array}\right]\left[\begin{array}{rrr}
1 & 2 &-1 \\
-3 & -5 & 0 \\
4 & 6 & 1
\end{array}\right]=\left[\begin{array}{rrr}
-6 & -6 & -5 \\-5 & -9 & 0 \\21 & 32 & 3
\end{array}\right]\]
Hence, $\{\Vec{u_1},\Vec{u_2},\Vec{u_3}\}=\left\{\left[\begin{array}{r}-6\\-5\\21\end{array}\right],\left[\begin{array}{r}-6\\-9\\32\end{array}\right],\left[\begin{array}{r}-5\\0\\3\end{array}\right]\right\}$
\item Similar to (1), we have
\[\left[\begin{array}{rrr}\Vec{w_1}&\Vec{w_2}&\Vec{w_3}\end{array}\right]P=\left[\begin{array}{rrr}\Vec{v_1}&\Vec{v_2}&\Vec{v_3}\end{array}\right]\]
\[\left[\begin{array}{rrr}\Vec{w_1}&\Vec{w_2}&\Vec{w_3}\end{array}\right]=\left[\begin{array}{rrr}\Vec{v_1}&\Vec{v_2}&\Vec{v_3}\end{array}\right]P^{-1}\]
\[=\left[\begin{array}{rrr}-2&-8&-7\\2&5&2\\3&2&6\end{array}\right]
\left[\begin{array}{rrr}5&8&5 \\-3&-5&-3 \\-2&-2&-1\end{array}\right]
=\left[\begin{array}{rrr}28&38&21 \\-9&-13&-7 \\-3&2&3\end{array}\right]\]
Hence, $\{\Vec{w_1},\Vec{w_2},\Vec{w_3}\}=\left\{\left[\begin{array}{r}28\\-9\\-3\end{array}\right],\left[\begin{array}{r}38\\-13\\2\end{array}\right],\left[\begin{array}{r}21\\-7\\3\end{array}\right]\right\}$
\end{enumerate}
\subsection*{Question 3.}
Let $\mathcal{B}=\{\Vec{b_1},\Vec{b_2}\}$, $\mathcal{C}=\{\Vec{c_1},\Vec{c_2}\}$ and $\mathcal{D}=\{\Vec{d_1},\Vec{d_2}\}$ be bases for a two dimensional vector space. Write an equation that relates the matrices $P_{\mathcal{B}\to\mathcal{C}}$, $P_{\mathcal{C}\to\mathcal{D}}$ and $P_{\mathcal{B}\to\mathcal{D}}$. Justify your answer. 
\subsection*{Solution 3.}
Matrix $P_{\mathcal{B}\to\mathcal{C}}$ converts a vector represented by basis $\mathcal{B}$ to basis $\mathcal{C}$. Matrix $P_{\mathcal{C}\to\mathcal{D}}$ converts a vector represented by basis $\mathcal{C}$ to basis $\mathcal{D}$. 
Matrix $P_{\mathcal{B}\to\mathcal{D}}$ converts a vector represented by basis $\mathcal{B}$ to basis $\mathcal{D}$.
Hence, to convert a vector represented by basis $\mathcal{B}$ to $\mathcal{D}$, we can alternatively convert it to representation by basis $\mathcal{C}$ then to $\mathcal{D}$. Therefore,
\[P_{\mathcal{B}\to\mathcal{D}}=P_{\mathcal{C}\to\mathcal{D}}P_{\mathcal{B}\to\mathcal{C}}\]
\subsection*{Question 4.}
Calculate the determinants by cofactor expansion:
\[\left\vert \begin{array}{rrr}
2 & 3 & -3 \\ 4 & 0 & 3 \\ 6 & 1 & 5
\end{array}\right\vert, \left\vert\begin{array}{rrrrr}
4 & 0 & -7 & 3 & -5 \\0 &0&2&0&0\\7&3&-6&4&-8\\5&0&5&2&-3\\0&0&9&-1&2 
\end{array}\right\vert.\]
\subsection*{Solution 4.}
\[\left\vert \begin{array}{rrr}
2 & 3 & -3 \\ 4 & 0 & 3 \\ 6 & 1 & 5
\end{array}\right\vert=-4\left\vert \begin{array}{rr}
3 & -3  \\ 1 & 5\end{array}\right\vert -3\left \vert \begin{array}{rr}
2 & 3  \\ 6 & 1\end{array}\right\vert=-72+48=-24\]
\[\left\vert\begin{array}{rrrrr}
4 & 0 & -7 & 3 & -5 \\0 &0&2&0&0\\7&3&-6&4&-8\\5&0&5&2&-3\\0&0&9&-1&2 
\end{array}\right\vert=-2\left\vert\begin{array}{rrrr}
4 & 0 & 3 & -5 \\7&3&4&-8\\5&0&2&-3\\0&0&-1&2 
\end{array}\right\vert=-6\left\vert\begin{array}{rrr}
4 &  3 & -5 \\5&2&-3\\0&-1&2 
\end{array}\right\vert\]
\[=-6\left(\left\vert\begin{array}{rr}
4 & -5 \\5&-3\end{array}\right\vert+2\left\vert\begin{array}{rr}
4 &  3  \\5&2\end{array}\right\vert\right)=-6(13-14)=6\]
\subsection*{Question 5.}
What is the determinant of an elementary row replacement matrix? What is the determinant of an elementary scaling matrix with k on the diagonal?
\subsection*{Solution 5.}
A row replacement matrix $E$ is a lower triangular matrix with all 1 in the entries of the main diagonal, hence $\det{E}=1$.\newline
A scaling matrix transforming vectors in $\mathbf{R}^n$ has the form of $\left[\begin{array}{rrrr}k \\&k\\&&\ddots\\&&&k\end{array}\right]_{n\times n}$,its determinant is $k^n$.
\subsection*{Question 6.}
Combine the methods of row reduction and cofactor expansion to calculate the determinants.
\[\left\vert\begin{array}{rrrr}
2 & 5 & 4 & 1 \\
4 & 7 & 6 & 2 \\
6 & -2 & -4 & 0\\
-6 & 7 & 7 & 0
\end{array}\right\vert,
\left\vert\begin{array}{rrrr}
1 & 5 & 4 & 1 \\
0 & -2 & -4 & 0 \\
3 & 5 & 4 & 1 \\
-6 & 5 & 5 & 0
\end{array}\right\vert\]
\subsection*{Solution 6.}
\[\left\vert\begin{array}{rrrr}
2 & 5 & 4 & 1 \\
4 & 7 & 6 & 2 \\
6 & -2 & -4 & 0\\
-6 & 7 & 7 & 0
\end{array}\right\vert
=
\left\vert\begin{array}{rrrr}
2 & 5 & 4 & 1 \\
0 & -3 & -2 & 0 \\
6 & -2 & -4 & 0\\
-6 & 7 & 7 & 0
\end{array}\right\vert
=-
\left\vert\begin{array}{rrr}
0 & -3 & -2 \\
6 & -2 & -4 \\
0 & 5 & 3 
\end{array}\right\vert
=6\left\vert\begin{array}{rr}
 -3 & -2 \\
 5 & 3 
\end{array}\right\vert=6
\]
\[
\left\vert\begin{array}{rrrr}
1 & 5 & 4 & 1 \\
0 & -2 & -4 & 0 \\
3 & 5 & 4 & 1 \\
-6 & 5 & 5 & 0
\end{array}\right\vert
=
\left\vert\begin{array}{rrrr}
1 & 5 & 4 & 1 \\
0 & -2 & -4 & 0 \\
2 & 0 & 0 & 0 \\
-6 & 5 & 5 & 0
\end{array}\right\vert
=-
\left\vert\begin{array}{rrr}
0 & -2 & -4 \\
2 & 0 & 0  \\
-6 & 5 & 5 
\end{array}\right\vert
=2(-10+20)=20
\]
\end{document}