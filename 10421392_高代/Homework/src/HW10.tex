\documentclass{article}
\usepackage[english]{babel}
\usepackage[a4paper,top=2.54cm,bottom=2.54cm,left=2.54cm,right=2.54cm,marginparwidth=1.75cm]{geometry}
\usepackage{amsmath}
\usepackage{graphicx}
\usepackage{amsfonts}
\usepackage{amssymb}
\usepackage{enumerate}
\usepackage{enumitem}
\usepackage[colorlinks=true, allcolors=blue]{hyperref}
\usepackage{graphicx}
\usepackage[export]{adjustbox}
\usepackage{multirow}
\usepackage{mathtools}
\usepackage{MnSymbol}%
\usepackage{wasysym}%
\title{Topics in Linear Algebra: Homework 10}
\begin{document}
\maketitle
\subsection*{Solution 1.10.1.}
\begin{enumerate}
\item 
\[T^{11} = T^{22} =  -\frac{1}{\sqrt{2}} -1, T^{12} = T^{21} = -1\]
\item

Indeed, by observation, the two columns in $\mathrm{Rot}(\pi/4)$ are possible to be $x$ and $y$, so let 
\[\mathbf{x} = \frac{1}{\sqrt{2}}\left[\begin{array}{c} 1 \\ 1 \end{array}\right],\mathbf{y} = \frac{1}{\sqrt{2}}\left[\begin{array}{c} -1 \\ 1 \end{array}\right] \]
Then, 
\[\mathbf{x} \otimes \mathbf{x}\]
has all entries $1/2$, and 
\[\mathbf{y} \otimes \mathbf{y}\]
has entry $1/2$ when having equal indices, and  $-1/2$ when having distinct indices. \\
Immediately
\[T = -(\sqrt{2}+1)\mathbf{x} \otimes \mathbf{x} - \mathbf{y} \otimes \mathbf{y}\]
\item
By observing the result of (2) and the definiton of $T$, the resultant magnitude of force in $\frac{1}{\sqrt{2}}\left[\begin{array}{c} 1 \\ 1\end{array}\right]$ direction is larger than that of $\frac{1}{\sqrt{2}}\left[\begin{array}{c} -1 \\ 1\end{array}\right]$ direction, so the direction of major axis is
$\frac{1}{\sqrt{2}}\left[\begin{array}{c} -1 \\ 1\end{array}\right]$, while the direction of minor is $\frac{1}{\sqrt{2}}\left[\begin{array}{c} 1 \\ 1\end{array}\right]$
\item
As the force are "normal" to the circle, 
\[T = \sum_{k=1} ^ r c_k \mathbf{x}_k \otimes \mathbf{x}_k\]
where $\mathbf{x}_k \in R^2$, $c_k\in R$, $||\mathbf{x}_k||=1$(in different directions), $c_k < 0,$ $k = 1,2, \cdots, r.$ \\
(Here only only consider forces on upper half of the circle, and the force on lower part are moved to the position $\pi$ rad apart of the original position of the circle, with direcction reversed). 
Let $\mathbf{x_k} = a_ki + b_kj, $ then
Matrix form of $T_{ij}$ is $T_j ^ i$
\[T_j ^ i = \left[ \begin{array}{cc}
\sum_{k=1} ^ r c_k (a_k)^2 & \sum_{k=1} ^ r c_k a_kb_k  \\
\sum_{k=1} ^ r c_k a_kb_k & \sum_{k=1} ^ r c_k (b_k)^2
\end{array}\right] \]
By Cauchy-Schwarz inequality, 
\[\mathrm{det}(T_j^i) = \left(\sum_{k=1} ^ r c_k (a_k)^2\right)\left(\sum_{k=1} ^ r c_k (b_k)^2\right) - \left(\sum_{k=1} ^ r c_k a_kb_k\right)^2\]
\[ = \left(\sum_{k=1} ^ r (\sqrt{-c_k}a_k)^2\right)\left(\sum_{k=1} ^ r (\sqrt{-c_k}b_k)^2\right) - \left(\sum_{k=1} ^ r (\sqrt{-c_k}a_k)(\sqrt{-c_k}b_k)\right)^2 \geq 0\]
Equality holds iff 
\[\frac{\sqrt{-c_k}a_k}{\sqrt{-c_k}b_k} = C\]
However, it is not possible, unless $r=1$. \\
And trivially, $\mathrm{tr}(T_j^i) <0$, so the matrix is negative semi-definite.
It is negative definite iff $r>1$. Moreover, the matrix is symmetric, when the matrix representation is not symmetric, or when it is not negative semi-definite, there must be force that is not perpendicularly towards the center of circle.\\
For instance, $T^{ij} = \delta ^{ij}$ is not possible.
\end{enumerate}
\subsection*{Solution 1.10.2.}
\begin{enumerate}
\item $\forall \alpha \in V^*, v \in V$, 
\[\alpha_{\mathcal{B}}(v_{\mathcal{B}}) = \alpha_{\mathcal{C}}(v_{\mathcal{C}}) = \alpha_{\mathcal{C}}(Mv_{\mathcal{B}})\]
\[\Leftrightarrow \alpha_{\mathcal{B}} = \alpha_{\mathcal{C}}M \Leftrightarrow \alpha_{\mathcal{C}} =  \alpha_{\mathcal{B}} M ^{-1}\]
\item

\[T(v,w) = (\sum_{i,j} x_{ij} b_i ^* \otimes b_j ^*)(u,v) = \sum_{i,j} x_{ij} b_i ^*(v)b_j^*(w)= x_{ij} v_{\mathcal{B}} ^ i w_{\mathcal{B}} ^ j = v_{\mathcal{B}} ^ i \delta_{ik} x_j ^ k w_{\mathcal{B}} ^ j = v_{\mathcal{B}}^T T_{\mathcal{B}} w_{\mathcal{B}}\]
\item
Replacing $x$ and $b$ to $y$ and $c$ in the proof above immediately yields
\[v_{\mathcal{C}}^T T_{\mathcal{C}} w_{\mathcal{C}}=T(u,v)=v_{\mathcal{B}}^T T_{\mathcal{B}} w_{\mathcal{B}}\]
Moreover, $v_\mathcal{C} = Mv_{\mathcal{B}}$, therefore
\[v_{\mathcal{B}}^T M^T T_{\mathcal{C}} Mw_{\mathcal{B}}=v_{\mathcal{B}}^T T_{\mathcal{B}} w_{\mathcal{B}}\]
for any $v,w$, and hence
\[M^T T_{\mathcal{C}} M = T_{\mathcal{B}} \Rightarrow T_{\mathcal{C}} = (M^T)^{-1}T_{\mathcal{B}} M ^{-1}\]
\item From 1, $\alpha_{\mathcal{B}} = \alpha_{\mathcal{C}}M$, so
\[ \alpha_{\mathcal{C}} T_{\mathcal{C}} \beta_{\mathcal{C}} ^T= T(\alpha, \beta)  = \alpha_{\mathcal{B}} T_{\mathcal{B}} \beta_{\mathcal{B}} ^T =\alpha_{\mathcal{C}}M T_{\mathcal{B}} M ^ T\beta_{\mathcal{C}} ^T\]
for any $\alpha, \beta$, and similarly
\[T_{\mathcal{C}}=MT_{\mathcal{B}}M^T\]
\item Similar to above,
\[\alpha_{\mathcal{C}}T_{\mathcal{C}} v_{\mathcal{C}}=T(\alpha, v) = \alpha_{\mathcal{B}}T_{\mathcal{B}} v_{\mathcal{B}} = \alpha_{\mathcal{C}} M T_{\mathcal{B}} M^{-1} v_{\mathcal{C}} \]
So
\[T_{\mathcal{C}} = MT_{\mathcal{B}}M ^{-1}\]
\end{enumerate}
\subsection*{Solution 1.10.3.}
Gradient:
\[\nabla f = 2xi+2yj+2zk\]
New function:
\[f_{\mathrm{new}}(x+y, y+z, z) = x^2+y^2+z^2\]
Let $u = x+y, v= y+z$, then 
\[f_{\mathrm{new}}(u,v,z) = (u-v+z)^2+(v-z)^2 + z^2 \]
\[= u^2+v^2+z^2-2uv-2vz+2uz+v^2+z^2-2vz+z^2\]
\[=u^2+2v^2+3z^2 -2uv-4vz+2zu\]
\[f_{\mathrm{new}}(x,y,z) = x^2+2y^2+3z^2-2xy-4yz+2zx\]
Gradient of new function:
\[\nabla f_{\mathrm{new}} = (2x-2y+2z)i+(4y-2x-4z)j+(6z-4y+2x)k\]
\[(M ^{-1})^T = ((I+N)^{-1})^T = (I-N+N^2) ^ T = \left[\begin{array}{ccc}
1  \\ -1 & 1 \\ 1 & -1 & 1 \end{array}\right]\]
Verification:
\[\text{R.H.S.} = (M ^{-1})^T(\nabla f(x,y,z) = 2xi + (-2x+2y)j + (2x-2y+2z)k\]
\[\text{L.H.S.} = \nabla f(x+y, y+z, z) = (2(x+y)-2(y+z)+2z)i + (4(y+z)-2(x+y)-4z)j + (6z-4(y+z)+2(x+y))k\]
\[=2xi +(2y-2x)j + (2z-2y+2x)k\]
So L.H.S = R.H.S. 
\end{document}