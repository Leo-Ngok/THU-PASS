\documentclass{article}
\usepackage[english]{babel}
\usepackage[a4paper,top=2.54cm,bottom=2.54cm,left=2.54cm,right=2.54cm,marginparwidth=1.75cm]{geometry}
\usepackage{amsmath}
\usepackage{graphicx}
\usepackage{amsfonts}
\usepackage{amssymb}
\usepackage{enumerate}
\usepackage{enumitem}
\usepackage[colorlinks=true, allcolors=blue]{hyperref}
\usepackage{graphicx}
\usepackage[export]{adjustbox}
\usepackage{multirow}
\usepackage{mathtools}
\usepackage{MnSymbol}%
\usepackage{wasysym}%
\title{Topics in Linear Algebra: Homework 4}
\begin{document}
\maketitle
\subsection*{Solution 1.4.1.}
\begin{enumerate}
    \item 
    $B=\left[
    \begin{array}{cc}
    I & X \\
         & I
    \end{array}
    \right]$, where $X$ satisfies 
    \[\left[
    \begin{array}{cc}
    1 & 2 \\
      & 1
    \end{array}
    \right]X-X\left[
    \begin{array}{cc}
    3 & 5 \\
      & 4
    \end{array}
    \right]=\left[
    \begin{array}{cc}
    1 & 2 \\
    3 & 4
    \end{array}
    \right]\]
    Let $X=\left[
    \begin{array}{cc}
    a & b \\
    c & d
    \end{array}
    \right]$, then
    \[\left\{\begin{array}{lcc}
    -2a+2c & = & 1 \\
    -5a-3b+2d  & = & 2\\
    -2c & = & 3\\
    -5c-3d & = & 4
    \end{array}\right.\]
    So $B=\left[
    \begin{array}{cccc}
    1 & 0 & -2 & 31/9 \\
    0 & 1 & -3/2 & 7/6 \\
    &&1\\
    &&&1
    \end{array}
    \right]$
    \item 
    \[\mathcal{B}=\left\{
    \left[
    \begin{array}{c} -4\\-3\\2\\0
    \end{array}
    \right],
    \left[
    \begin{array}{c} 62\\21\\0\\18
    \end{array}
    \right]
    \right\}\]
    is a basis that span $V_3+V_4$.
\end{enumerate}
\subsection*{Solution 1.4.2.}
\begin{enumerate}
    \item 
    \[\left[
    \begin{array}{cccc}
    I\\&&I\\&I\\&&&I
    \end{array}
    \right]\left[
    \begin{array}{cccc}
    X&&I\\&Y&&I\\&&X\\&&&Y
    \end{array}
    \right]=\left[
    \begin{array}{cccc}
    X&&I\\&&X\\&Y&&I\\&&&Y
    \end{array}
    \right]
    \]
    \[\left[
    \begin{array}{cccc}
    X&I\\&X\\&&Y&I\\&&&Y
    \end{array}
    \right]\left[
    \begin{array}{cccc}
    I\\&&I\\&I\\&&&I
    \end{array}
    \right]=\left[
    \begin{array}{cccc}
    X&&I\\&&X\\&Y&&I\\&&&Y
    \end{array}
    \right]\]
    Hence $\left[
    \begin{array}{cccc}
    X&&I\\&Y&&I\\&&X\\&&&Y
    \end{array}
    \right]$ and $\left[
    \begin{array}{cccc}
    X&I\\&X\\&&Y&I\\&&&Y
    \end{array}
    \right]$ are similar.
    \item
    \[p_B(x)=(3-x)(4-x)=12-7x+x^2,\]
    \[p_A(x)=(3-x)^2(4-x)^2=144-168x+73x^2-14x^3+x^4\]
    \item 
    \[A=\left[
    \begin{array}{cccc}
    &1&1\\&&&1\\&&&1\\&&&0
    \end{array}
    \right]\]
    A has two pivot columns, rank($A)=2$.\\
    \[A^2=\left[
    \begin{array}{cccc}
    0&0&0&2\\&&&0\\&&&0\\&&&0
    \end{array}
    \right],\]
    So rank($A^2)=1$, rank($A^3)=$rank($A^4)=0$.
    \[J=\left[
    \begin{array}{cccc}
    0&1\\&0&1\\&&0\\&&&0
    \end{array}
    \right]\]
    \item $p_B(x)=x^2,p_A(x)=x^3$
    \item If 
    \[p_B(x)=\prod_{i=1}^s (x-\lambda_i I)^{m_i},\]
    where $m_i$ is the algebraic multiplicity of $\lambda_i$, and $s$ is number of distinct eigenvalues of $B$, then
    \[p_A(x)=p_B(x)\cdot\prod_{i=1}^s(x-\lambda_i I)^{\mathrm{Dim(Ker(}A-\lambda_i I))}\]
\end{enumerate}
\subsection*{Solution 1.4.3.}
\begin{enumerate}
    \item $\forall a:a\in R,$ 
    \[L(aI)=aX-Xa=aX-aX=0\]
    \item
    \[L(X)Y+XL(Y)=(AX-XA)Y+X(AY-YA)=AXY-XAY+XAY-XYA\]
    \[=A(XY)-(XY)A=L(XY)\]
    \item
    Here we prove the following lemma first:
    \[S(n): L(X^n)=n\cdot L(X)\cdot X^{n-1}\]
    Showing that $S(n)$ holds $\forall n\in Z^+$.\\
    $S(1)$ is trivial.\\
    Suppose $S(k)$ holds, then for $S(k+1)$,\\
    \[L(X^{k+1})=XL(X^k)+L(X)X^k\]
    \[=X(kL(X)X^{k-1})+L(X)X^k=kL(X)X^k+L(X)X^k=(k+1)L(X)X^k,\]
    So $S(k)\Rightarrow S(k+1)$. By first principle of induction, $(\forall n)(n\in Z^+\Rightarrow S(n))$.\\
    $p(x)$ is a polynomial, i.e. it is analytic, and has finite terms in its Taylor's expansion, so W.L.O.G. let 
    $p(x)=\sum_{i=0}^n a_ix^i$
    \[L(p(X))=L(\sum_{i=0}^n a_i\cdot X^i)=\sum_{i=0}^n a_i\cdot L(X^i)=\sum_{i=1}^n a_i\cdot L(X^i)=\sum_{i=1}^n a_i\cdot i\cdot L(X) \cdot X^{i-1} = L(X) p'(X)\]
    \item Given $L(X)=I$, then for any polynomial $p$,
    \[L(p(X))=L(X)p'(X)=p'(X),\]
    By choosing $p$ as the minimal polynomial of $X$, deg $p>0$, then
    \[p'(X)=L(p(X))=L(0)=0,\]
    So $X$ is one of the zeros of  $p'$, that deg $p'<$ deg $p$, contradicting the minimality of $p$, hence $L(X)=I$ is not possible.  
    \item Given that $A$ is diagonalizable with distinct eigenvalues, so W.L.O.G. let $A=P\Lambda P^{-1}$, then
    \[ L(X)=0 \Leftrightarrow AX=XA \Leftrightarrow P\Lambda P^{-1}X = X P\Lambda P^{-1}\Leftrightarrow \Lambda P^{-1} XP=P^{-1}XP\Lambda\]
    So $L(X)=0$ iff $[P^{-1}XP,\Lambda]=0$ iff $P^{-1}XP$ is diagonal iff $A$ and $X$ shares the same eigenvectors, and hence Dim(Ker($L))=n$.
    \item Let $t_1,t_2,t_3$ are all distinct. Then $\forall X:X\in M_n(R),$
    \[L(X)=\left[\begin{array}{ccc}
    t_1\\&t_2\\&&t_3
    \end{array}\right]X
    -X\left[\begin{array}{ccc}
    t_1\\&t_2\\&&t_3
    \end{array}\right]\]
    \[
    =\left[\begin{array}{ccc}
    t_1\\&t_2\\&&t_3
    \end{array}\right]
    \left[\begin{array}{ccc}
    a_{11}&a_{12}&a_{13}\\a_{21}&a_{22}&a_{23}\\a_{31}&a_{32}&a_{33}
    \end{array}\right]-
    \left[\begin{array}{ccc}
    a_{11}&a_{12}&a_{13}\\a_{21}&a_{22}&a_{23}\\a_{31}&a_{32}&a_{33}
    \end{array}\right]
    \left[\begin{array}{ccc}
    t_1\\&t_2\\&&t_3
    \end{array}\right]
    \]
    \[
    =\left[\begin{array}{ccc}
    0&(t_1-t_2)a_{12}&(t_1-t_3)a_{13}\\
    (t_2-t_1)a_{21}&0&(t_2-t_3)a_{23}\\
    (t_3-t_1)a_{31}&(t_3-t_2)a_{32}&0
    \end{array}\right]
    \]
    Ran($L)$ are hollow matrices.
    For instance, 
    \[A=\left[\begin{array}{ccc}
    1\\&2\\&&3
    \end{array}\right]\]
\end{enumerate}
\end{document}