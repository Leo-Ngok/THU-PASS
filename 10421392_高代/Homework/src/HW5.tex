\documentclass{article}
\usepackage[english]{babel}
\usepackage[a4paper,top=2.54cm,bottom=2.54cm,left=2.54cm,right=2.54cm,marginparwidth=1.75cm]{geometry}
\usepackage{amsmath}
\usepackage{graphicx}
\usepackage{amsfonts}
\usepackage{amssymb}
\usepackage{enumerate}
\usepackage{enumitem}
\usepackage[colorlinks=true, allcolors=blue]{hyperref}
\usepackage{graphicx}
\usepackage[export]{adjustbox}
\usepackage{multirow}
\usepackage{mathtools}
\usepackage{MnSymbol}%
\usepackage{wasysym}%
\title{Topics in Linear Algebra: Homework 5}
\begin{document}
\maketitle
\subsection*{Solution 1.5.1.}
First, examine the differentiability of $f$.\\
For $f: C\to C$, $f = u + iv$ is differentiable at $z_0=x_0+iy_0$ iff 
\[\left.\frac{\partial u}{\partial x}\right|_{x=x_0}= \left.\frac{\partial v}{\partial y}\right|_{y=y_0}, \left.\frac{\partial u}{\partial y}\right|_{y=y_0}=-\left.\frac{\partial v}{\partial x}\right|_{x=x_0}\]
For $f(z)=z|z|$, $u(x,y)=x\sqrt{x^2+y^2}$,  $v(x,y)=y\sqrt{x^2+y^2}$, then \\
\[ u_x = \sqrt{x^2+y^2}+\frac{x^2}{\sqrt{x^2+y^2}}, u_y=\frac{xy}{\sqrt{x^2+y^2}}\]
\[v_x= \frac{xy}{\sqrt{x^2+y^2}}, v_y=\sqrt{x^2+y^2}+\frac{y^2}{\sqrt{x^2+y^2}}.\]
Solving $u_x=v_y$, $v_x=-u_y$ yields 
\[\left\{\begin{array}{l}x^2=y^2\\2xy=0\\x^2+y^2>0\end{array}\right.\]
However, no such tuple $(x_0,y_0)$ satisfies all three equations above, and hence $f$ is nowhere complex differentiable.\\
But if it is the case $f|_R: R\to R$, then it is real differentiable everywhere.\\

$f(x)=x|x|=x\sqrt{x^2}$, then $f'(x)=\sqrt{x^2}+x\cdot \frac{x}{\sqrt{x^2}}=2|x|$
\begin{enumerate}
    \item 
    \[A_t=
    \left[\begin{array}{cc}
    1 & 1 \\
         & 1+t
    \end{array}\right] =
    \left[\begin{array}{cc}
    1 & -x \\
         & 1
    \end{array}\right]
    \left[\begin{array}{cc}
    1 &  \\
         & 1+t
    \end{array}\right]
    \left[\begin{array}{cc}
    1 & -x \\
         & 1
    \end{array}\right]
    \]
    So $1 = -xt$, $x=-1/t$.
    \[f(A_t)=
    \left[\begin{array}{cc}
    1 & 1/t \\
         & 1
    \end{array}\right]
    \left[\begin{array}{cc}
    1 &  \\
         & (1+t)^2
    \end{array}\right]
    \left[\begin{array}{cc}
    1 & -1/t \\
         & 1
    \end{array}\right]
    =
    \left[\begin{array}{cc}
    1 & t+2 \\
         & (1+t)^2
    \end{array}\right], \forall t: t\geq -1
    \]
    So
    \[\lim_{t\to 0} f(A_t) = \lim_{t\to 0}\left[\begin{array}{cc}
    1 & t+2 \\
         & (1+t)^2
    \end{array}\right]=\left[\begin{array}{cc}
    1 & 2 \\
         & 1
    \end{array}\right] \]
    \item $f$ is real differentiable, but not complex differentiable, which is obviously not analytic, so $f$ cannot equal to its Taylor series, hence $f$ is not defined for matrices with complex eigenvalues.\\
    Moreover, 
    \[p_{A_t}(\lambda) = \det\left(\left[\begin{array}{cc}
    1-\lambda & 1 \\
    -t^2 & 1-\lambda
    \end{array}\right]\right)=(1-\lambda)^2+t^2\]
    so for $A_t$, $\lambda = 1 \pm it$.\\
    As the limit is taking values from the punctured neighborhood of zero, the limit does not exist, so $f$ is not defined.
\end{enumerate}

\subsection*{Solution 1.5.2.}
\begin{enumerate}
    \item $\sin(z)=\frac{e^{iz}-e^{-iz}}{2i}$, so
    \[\sin(tA)=\frac{e^{itA}-e^{-itA}}{2i},\frac{d}{dt}\sin(tA)=\frac{iAe^{itA}-(-iA)e^{-itA}}{2i}=\frac{A(e^{itA}+e^{-itA})}{2}=A\cos(tA)\]
    \item W.L.O.G., assume $f$ is analytic at $z = 0$ for $f:\mathbf{C}\to \mathbf{C}$.\\
    Then, 
    \[f(z) = \sum_{k=0}^{+\infty} a_k z^k,\]
    where 
    \[a_k = \frac{1}{2\pi i}\oint_{|z|=\epsilon} \frac{f(z)dz}{z^{n+1}}\]
    Before seeking for the solution, first prove a lemma:
    \[S(n):\left[\begin{array}{cc}
    2A & A \\
     & 2A
    \end{array}\right]^n = \left[\begin{array}{cc}
    (2A)^n & n\cdot 2^{n-1} A^n \\
     & (2A)^n
    \end{array}\right],\forall n: n\in \mathbf{Z}^+\]
    $S(1)$ is trivial.\\
    Assume $S(k)$ holds, then for $S(k+1)$,
    \[\left[\begin{array}{cc}
    2A & A \\
     & 2A
    \end{array}\right]^{k+1} = 
    \left[\begin{array}{cc}
    2A & A \\
     & 2A
    \end{array}\right]^k 
    \left[\begin{array}{cc}
    2A & A \\
     & 2A
    \end{array}\right]
    =
    \left[\begin{array}{cc}
    (2A)^k & k\cdot 2^{k-1} A^k \\
     & (2A)^k
    \end{array}\right]
    \left[\begin{array}{cc}
    2A & A \\
     & 2A
    \end{array}\right]\]
    \[=\left[\begin{array}{cc}
    (2A)^{k+1} & (2^k+k\cdot 2^k)A^{k+1} \\
     & (2A)^{k+1}
    \end{array}\right]=
    \left[\begin{array}{cc}
    (2A)^{k+1} & (k+1)\cdot 2^k A^{k+1} \\
     & (2A)^{k+1}
    \end{array}\right],S(k)\Rightarrow S(k+1)\]
    By first principle of induction, $\forall n:n\in \mathbf{Z}^+\Rightarrow S(n)$. $\blacksquare$\\
    So 
    \[f\left(\left[\begin{array}{cc}
    2A & A \\
     & 2A
    \end{array}\right]\right)= \sum_{k=0}^{+\infty} a_k \left[\begin{array}{cc}
    2A & A \\
     & 2A
    \end{array}\right]^k = \sum_{k=0}^{+\infty}\left[\begin{array}{cc}
    a_k (2A)^k & a_k\cdot k\cdot 2^{k-1} A^k \\
     & a_k (2A)^k
    \end{array}\right] \]
    Let $X=BJ B^{-1}$, then for  $p(x)=x^n$, $p'(X)=Bp'(J)B^{-1}$, $p'(J)= n J^{n-1}$, so $p'(X)=nX^{n-1}$
    \[\Rightarrow f'(X)=\sum_{k=1}^{+\infty} a_k\cdot k\cdot X^{k-1}\]
    \[\Rightarrow f\left(\left[\begin{array}{cc}
    2A & A \\
     & 2A
    \end{array}\right]\right) =\left[\begin{array}{cc}
    f(2A) & f'(A)A \\
     & f(2A)
    \end{array}\right]\Rightarrow B=f'(A)\cdot A.\]
    \item It can be disproved by letting $f(x)=x^2$, $A=\left[\begin{array}{cc}
    1 & 1 \\
     & 1
    \end{array}\right]$ and $B = \left[\begin{array}{cc}
    2 \\
     & 3
    \end{array}\right]$
    \[f'(A)B=\left(\left[\begin{array}{cc}
    f'(1) & f''(1) \\
     & f'(1)
    \end{array}\right]\right)\left[\begin{array}{cc}
    2  \\
     & 3
    \end{array}\right] = \left[\begin{array}{cc}
    4 & 6 \\
     & 6
    \end{array}\right]\]
    \[f'\left(\left[\begin{array}{cc}
    1+2t & 1 \\
     & 1+3t
    \end{array}\right]\right)=\left[\begin{array}{cc}
    1 & -1/t \\
     & 1
    \end{array}\right]\left[\begin{array}{cc}
    2+4t  \\
     & 2+6t
    \end{array}\right]\left[\begin{array}{cc}
    1 & 1/t \\
     & 1
    \end{array}\right]=\left[\begin{array}{cc}
    2+4t & -2 \\
     & 2+6t
    \end{array}\right]\]
    Plugging in $t=0$ yields $\left[\begin{array}{cc}
    2 & -2 \\
     & 2
    \end{array}\right].$\\
    Actually, the proposition holds only when $[A,B]=0$. In this case, binomial theorem is suitable for matrices $A$ and $B$.
    \[[f(A+tB)]' = \left(\sum_{n=0}^{+\infty}\sum_{k=0}^n a_n C_n^k A^{n-k}(Bt)^k \right)'=\sum_{n=0}^{+\infty}\sum_{k=0}^n a_n C_n^k A^{n-k}k B^kt^{k-1} \]
    Plugging in $t=0$,
    \[[f(A+tB)]'|_{t=0}=\sum_{n=1}^{+\infty} a_n C_n^1 A^{n-1} B=f'(A)\cdot B\]
\end{enumerate}

\subsection*{Solution 1.5.3.}
\begin{enumerate}
    \item Let $v$ be the common eigenvector of $A$ and $B$, $\lambda_A$ and $\lambda_B$ are the two eigenvalues associated with $v$ for $A$ and $B$.  $V_2$ is a matrix such that the column vectors are orthonormal basis of orthogonal complement of span of $v$. Then for $A$,
    \[\left[\begin{array}{cc}v&V_2
    \end{array}\right]^H A\left[\begin{array}{cc}v&V_2
    \end{array}\right]=\left[\begin{array}{cc}v&V_2
    \end{array}\right]^H\left[\begin{array}{cc}\lambda_Av &A V_2
    \end{array}\right]=\left[\begin{array}{cc}\lambda_A&v^HAV_2\\0&V_2^HAV_2
    \end{array}\right]\]
    Similarly, for $B$, 
    \[\left[\begin{array}{cc}v&V_2
    \end{array}\right]^H B\left[\begin{array}{cc}v&V_2
    \end{array}\right]=\left[\begin{array}{cc}\lambda_B&v^HBV_2\\0&V_2^HBV_2
    \end{array}\right]\]
    So $A_1=V_2^HAV_2$, $B_1=V_2^HBV_2$.\\
    \[A_1B_1=V_2^HAV_2V_2^HBV_2=V_2^HABV_2\]
    \[B_1A_1=V_2^HBV_2V_2^HAV_2=V_2^HBAV_2\]
    The above two equations are equal as $A$ commutes with $B$, i.e.
    \[[A,B]=0 \Rightarrow [A_1,B_1]=0\]
    \item From (1), $[A_1,B_1]=0$, so $A_1$, $B_1$ should have a common eigenvector. We may assume that it is one of the column vectors of $V_2$, otherwise $V_2$ can be reconstructed in the previous step to contain that common eigenvector. \\
    Moveover, when that common eigenvector is placed on the first column of $V_2$ in (1), then 
    \[A_1=\left[\begin{array}{cc}\lambda_{A_1}&v'^HBV_{22}\\0&V_{22}^HA_1V_{22}
    \end{array}\right],\] 
    where $v'$ is the common eigenvector of $A_1$ and $B_1$. Same for $B_1$.\\
    Hence we can iteratively apply that transformation to $(A_i)_{22}$ and $(B_i)_{22}$ for all $i$. \\
    In addition, for each iteration, the size of $A_i$, $B_i$ strictly decreases.\\
    Given that $A$ and $B$ are finite dimensional, the process of iteration eventually terminates. \\
    The iterations terminate when $i$ is such that $A_i$ and $B_i$ has dimension one. \\
    That gives
    \[\left[\begin{array}{cc}v&V_2
    \end{array}\right]^H A\left[\begin{array}{cc}v&V_2
    \end{array}\right]\]
    an upper triangular matrix. Same for $B$.  $\blacksquare$
\end{enumerate}
\end{document}