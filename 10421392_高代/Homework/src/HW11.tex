\documentclass{article}
\usepackage[english]{babel}
\usepackage[a4paper,top=2.54cm,bottom=2.54cm,left=2.54cm,right=2.54cm,marginparwidth=1.75cm]{geometry}
\usepackage{amsmath}
\usepackage{graphicx}
\usepackage{amsfonts}
\usepackage{amssymb}
\usepackage{enumerate}
\usepackage{enumitem}
\usepackage[colorlinks=true, allcolors=blue]{hyperref}
\usepackage{graphicx}
\usepackage[export]{adjustbox}
\usepackage{multirow}
\usepackage{mathtools}
\usepackage{MnSymbol}%
\usepackage{wasysym}%
\title{Topics in Linear Algebra: Homework 11}
\begin{document}
\maketitle
Note: $\mathrm{sgn}$ of a permutation is $(-1) ^\text{\{number of inversions of the permutation\}} $.
\subsection*{Solution 1.11.1.}
\begin{enumerate}
    \item 
    \[e^1 \otimes e^2(e_1\otimes e_2) = e^1(e_1)e^2(e_2)=1\]
    \item 
    \[\mathrm{Alt}(e^1\otimes e^2) (e_1\otimes e_2) = \frac{1}{2}(e^1\otimes e^2 - e^2\otimes e^1)(e_1\otimes e_2) = \frac{1}{2}\]
    \item
    \[e^1\otimes e^2 (\mathrm{Alt}(e_1\otimes e_2)) = e^1\otimes e^2(\frac{1}{2}(e_1\otimes e_2 - e_2\otimes e_1)) = \frac{1}{2}\]
    \item
    \[\mathrm{Alt}(e^1\otimes e^2)\mathrm{Alt}(e_1\otimes e_2) = \frac{1}{4}(e^1\otimes e^2 - e^2 \otimes e^1)(e_1\otimes e_2 - e_2\otimes e_1) = \frac{1}{2}\]
    \item 
    \[e^1\wedge e^2 (\mathrm{Alt}(e_1\otimes e_2)) = 2\mathrm{Alt}(e^1\otimes e^2)\mathrm{Alt}(e_1\otimes e_2) = 1\]
    \item 
    \[e^1\wedge e^2(e_1\otimes e_2) = 2\mathrm{Alt}(e^1\otimes e^2) (e_1\otimes e_2) = 1\]
    \item
    \[e^1\otimes e^2 (e_1 \wedge e_2) = 2e^1\otimes e^2 (\mathrm{Alt}(e_1\otimes e_2)) = 1\]
    \item
    \[e^1 \wedge e^2 (e_1\wedge e_2) = 4\mathrm{Alt}(e^1\otimes e^2)\mathrm{Alt}(e_1\otimes e_2) = 2\]
    \item 
    \[\langle e_1\otimes e_2, e_1\otimes e_3 \rangle = \langle e_1,e_1\rangle \langle e_2,e_3\rangle = 0\]
    \[\langle u_1 \wedge v_1, u_2 \wedge v_2\rangle = \langle u_1 \otimes v_1 - v_1 \otimes u_1, u_2 \otimes v_2 - v_2 \otimes u_2\rangle\]
    \[=\langle u_1\otimes v_1, u_2\otimes v_2\rangle - \langle u_1\otimes v_1, v_2\otimes u_2\rangle - \langle v_1\otimes u_1, u_2\otimes v_2\rangle + \langle v_1\otimes u_1, v_2\otimes u_2\rangle \]
    \[=\langle u_1, u_2\rangle\langle v_1,v_2\rangle - \langle u_1, v_2\rangle\langle v_1,u_2\rangle - \langle v_1, u_2\rangle\langle u_1,v_2\rangle+\langle v_1, v_2\rangle\langle u_1,u_2\rangle\]
    \[=2(\langle u_1, u_2\rangle\langle v_1,v_2\rangle - \langle u_1, v_2\rangle\langle v_1,u_2\rangle)\]
    \[\langle e_1 \wedge e_2, e_1\wedge e_3\rangle = 2(\langle e_1,e_1\rangle \langle e_2, e_3\rangle - \langle e_1, e_3\rangle \langle e_1, e_2\rangle) = 0\]
    \item
    \[\langle e_1 \wedge e_2 \wedge e_3,e_1 \wedge e_2 \wedge e_3 \rangle \]
    \[ = \sum_{\sigma,\tau \in S_3} \mathrm{sgn}(\sigma)\mathrm{sgn}(\tau)\langle \sigma(e_{123}),\tau(e_{123})\rangle\]
    \[=\sum_{\sigma,\tau \in S_3} \mathrm{sgn}(\sigma)\mathrm{sgn}(\tau)\delta_{il}\delta_{jm}\delta_{ln}e_{\sigma(1)}^i e_{\sigma(2)}^j e_{\sigma(3)}^k e_{\tau(1)}^l e_{\tau(2)}^m e_{\tau(3)}^n\]
    Any term in this sum is 1 iff $\sigma = \tau$, and 0 otherwise. So,
    \[ = \sum_{\sigma,\tau \in S_3} \mathrm{sgn}(\sigma)\mathrm{sgn}(\tau)\langle \sigma(e_{123}),\tau(e_{123})\rangle\]
    \[=\sum_{\sigma\in S_3}\delta_{il}\delta_{jm}\delta_{ln}e_{\sigma(1)}^i e_{\sigma(2)}^j e_{\sigma(3)}^k e_{\sigma(1)}^l e_{\sigma(2)}^m e_{\sigma(3)}^n = 6\]
    \[\Rightarrow ||e_1 \wedge e_2 \wedge e_3|| = \sqrt{6}\]
    \[||\mathrm{Alt}(e_1\otimes e_2 \otimes e_3)|| = ||\frac{1}{3!} e_1 \wedge e_2 \wedge e_3|| = \frac{\sqrt{6}}{6}\]
    \[||\mathrm{Alt}(e_1\otimes e_2)\wedge e_3|| = \frac{1}{2!}||e_1 \wedge e_2 \wedge e_3|| = \frac{\sqrt{6}}{2}\]
\end{enumerate}
\subsection*{Solution 1.11.2.}
\begin{enumerate}
    \item In $\bigwedge_3 R^3$, $\forall u,v,w \in R^3$
    \[\mathrm{det}(u,v,w) = \sum _{\sigma \in S_3} \mathrm{sgn}(\sigma) u ^ {\sigma(1)} v ^ {\sigma(2)} w ^ {\sigma(3)}\]
    By definition of Levi-Civita notation, 
    \[e_{ijk} = \left\{\begin{array}{ll}
    0, & (i-j)(j-k)(k-i) = 0 \\
    \mathrm{sgn}(\sigma), & \text{otherwise}
    \end{array}\right.\]
    where for the latter case, in particular, $\sigma(1) = i, \sigma(2) = j, \sigma(3) = k$. That immediately yields
    \[\mathrm{det}(u,v,w) = e_{ijk} u^i v^j w^k = e_{ijk} e^i(u)e^j(v)e^k(w) = (e_{ijk} e^i \otimes e^j \otimes e^k)(u,v,w)\]
    \item $\forall u,v\in R^2, \alpha, \beta \in (R^2)^*$,
    \[e_{ij} e^{mn}u^i v^j \alpha_m \beta_n = e_{ij} u^iv^je^{mn} \alpha_m \beta_n = \mathrm{det}\left(\left[\begin{array}{cc}
    \alpha(u) & \alpha(v) \\
    \beta(u) & \beta(v)
    \end{array}\right]\right) = \alpha(u)\beta(v) - \alpha(v)\beta(u)\]
    \[(\delta_i^m\delta_j^n-\delta_i^n\delta_j^m)(u^iv^j\alpha_m\beta_n) = \delta_i^m u^i \alpha_m \delta_j^n v^j \beta_n -\delta_i^n u^i \beta_n \delta_j^m v^j \alpha_m = \alpha(u)\beta(v) - \beta(u)\alpha(v)\]
    \[\Rightarrow \delta_i^m\delta_j^n-\delta_i^n\delta_j^m = e_{ij} e^{mn}\]
    \item
    For a component of $e_{ijk} e^{imn}$ to be nonzero, only $j=m, k = n$ or $j = n, k = m$ is possible. \\
    For the former case, the parities of the two permutations are the same, which yields 1. \\
    For the latter case, the parities of the two permutations are different, which yields -1. \\
    Therefore, $e_{ijk} e^{imn} = \delta_j ^ m \delta_k^ n -\delta_j ^ n\delta _k ^m$     $\square$.
    \item Set $m=j$, then $e_{ijk} e^{ijn} = \delta_j ^ j \delta_k^ n -\delta_j ^ n\delta _k ^j = 3\delta_k^ n - \delta_k ^n = 2\delta_k^n$ \\
    Set $n = k$, then $e_{ijk} e^{ijk} = 2\delta_k^k = 6$
\end{enumerate}
\subsection*{Solution 1.11.3.}
\begin{enumerate}
    \item $\tau^2 = \frac{1}{2}(e_{1122}-e_{1221}-e_{2112}+e_{2211})$, 
    $$\langle a\otimes b\otimes c \otimes d, e\otimes f \otimes g \otimes h\rangle = \langle a\otimes b, e\otimes f\rangle \langle c\otimes d, g\otimes h\rangle = \langle a,e\rangle\langle b,f\rangle\langle c,g\rangle\langle d,h\rangle $$
    So expansion of the inner product by enumerating all combinations (through distributing properties) ignores non-identical terms.
    \[\langle \tau^2,\tau^2\rangle = \frac{1}{4}\left(
    \langle e_{1122}, e_{1122}\rangle +\langle e_{1221}, e_{1221}\rangle + \langle e_{2112}, e_{2112}\rangle +\langle e_{2211}, e_{2211}\rangle \right) = 1\Rightarrow ||\tau^2|| = 1\]
    Suppose $\tau = u\otimes v$. Then W.L.O.G. let $u = u^1e_1+u^2e_2, v=v^1e_1+v^2e_2$. 
    \[\tau = u\otimes v = (u^1e_1+u^2e_2)\otimes (v^1e_1+v^2e_2) = u^1v^1e_1\otimes e_1 + u^1v^2e_1\otimes e_2 + u^2v^1e_2\otimes e_1 + u^2v^2 e_2\otimes e_2\]
    Hence, $u^1v^1=u^2v^2=0,u^1v^2=\frac{1}{\sqrt{2}}, u^2v^1 = -\frac{1}{\sqrt{2}}$, and $0 = u^1v^1u^2v^2 = -\frac{1}{2}$, which is impossible. $\square$
    \item Trivially its matrix representation is $\left[\begin{array}{cc} 1 \\ & -1 \end{array}\right]$, so $M_A = e_1^T \otimes e_1^T - e_2^T \otimes e_2^T$.
    \item Its matrix representation is $
    \left[\begin{array}{cc} 1 & 1 \\ 1 & -1 \end{array}\right]
    \left[\begin{array}{cc} 1 \\ & -1 \end{array}\right]
    \left[\begin{array}{cc} 1 & 1\\ 1 & -1 \end{array}\right] ^ {-1} = \left[\begin{array}{cc}  & 1 \\ 1 \end{array}\right]$, \\
    $N_A = e_1^T \otimes e_2^T + e_2^T \otimes e_1^T$.
    \item Matrix multiplication is associative, and is distributive over addition, so the matrix representation of the sum of the two bilinear map is just the sum of the matrix: \\
    For $M_B$, its representation is 
    \[\frac{1}{\sqrt{2}} \left[\begin{array}{cc} 1 & 1 \\ 1 & -1 \end{array}\right]\]
    That matrix has determinant -1 and trace 0, so $\pm 1$ are the eigenvalues. \\
    For $N_B$, its representation is 
    \[\frac{1}{\sqrt{2}} \left[\begin{array}{cc} 1 & -1 \\ -1 & -1 \end{array}\right]\]
    That matrix has determinant -1 and trace 0, so $\pm 1$ are the eigenvalues.
    \item 
    \[M_A \otimes M_B + M_A\otimes N_B+N_A\otimes M_B - N_A\otimes N_B \]
    \[=\frac{1}{\sqrt{2}}(
    M_A\otimes M_A + M_A\otimes N_A + M_A\otimes M_A - M_A\otimes N_A + N_A\otimes M_A + N_A\otimes N_A - N_A\otimes M_A + N_A\otimes N_A )\]
    \[=\sqrt{2}(M_A\otimes M_A + N_A \otimes N_A) = \sqrt{2}((e^{11}-e^{22})\otimes(e^{11}-e^{22})+(e^{12}+e^{21})\otimes(e^{12}+e^{21}) )\]
    \[=\sqrt{2}(e^{1111}-e^{1122}-e^{2211}+e^{2222}+e^{1212}+e^{1221}+e^{2112}+e^{2121}),\]
    \[(M_A \otimes M_B + M_A\otimes N_B+N_A\otimes M_B - N_A\otimes N_B)(\tau^2)\]
    \[=\frac{\sqrt{2}}{2}(-e^{1122}(e_{1122})+e^{1221}(-e_{1221})+e^{2112}(-e_{2112})-e^{2211}(e_{2211})) = -2\sqrt{2}\]
    \item 
    Let $v = \cos \theta e_1 + \sin \theta e_2$, $w = \cos\phi e_1 + \sin\phi e_2$.
    \[v\otimes v = \cos^2\theta e_{11} + \cos\theta \sin\theta (e_{12}+e_{21}) + \sin^2 \theta e_{22}\]
    \[v\otimes v \otimes w \otimes w = \cos^2\theta \cos^2 \phi e_{1111} + \cos^2\theta \sin^2 \phi e_{1122} \]
    \[+ \cos\theta \sin\theta \cos \phi\sin\phi (e_{1212}+e_{1221}+e_{2112}+e_{2121}) + \sin^2\theta \cos^2 \phi e_{2211} + \sin^2 \theta \sin^2 \phi e_{2222} +\cdots\]
    The 8 other terms are neglected as they are irrelevant in calculation below.
    \[(M_A \otimes M_B + M_A\otimes N_B+N_A\otimes M_B - N_A\otimes N_B)(v\otimes v \otimes w \otimes w)\]
    \[=\sqrt{2}(\cos^2\theta \cos^2 \phi-\cos^2\theta \sin^2 \phi + 4\cos\theta \sin\theta \cos \phi\sin\phi -\sin^2\theta \cos^2 \phi + \sin^2 \theta \sin^2 \phi)\]
    \[=\sqrt{2}((\cos^2\theta-\sin^2\theta)(\cos^2\phi-\sin^2\phi)+ \sin(2\theta)\sin(2\phi))\]
    \[=\sqrt{2}(\cos(2\theta)\cos(2\phi)+\sin(2\theta)\sin(2\phi)) = \sqrt{2}\cos(2\theta - 2\phi),\]
    It is trivial that $-\sqrt{2}\leq \sqrt{2}\cos(2\theta - 2\phi) \leq \sqrt{2}.$  $\blacksquare$
\end{enumerate}
\end{document}