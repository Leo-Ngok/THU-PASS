\documentclass{article}
\usepackage[english]{babel}
\usepackage[a4paper,top=2.54cm,bottom=2.54cm,left=2.54cm,right=2.54cm,marginparwidth=1.75cm]{geometry}
\usepackage{amsmath}
\usepackage{graphicx}
\usepackage{amsfonts}
\usepackage{amssymb}
\usepackage{enumerate}
\usepackage{enumitem}
\usepackage[colorlinks=true, allcolors=blue]{hyperref}
\usepackage{graphicx}
\usepackage[export]{adjustbox}
\usepackage{multirow}
\usepackage{mathtools}
\usepackage{MnSymbol}%
\usepackage{wasysym}%
\title{Topics in Linear Algebra: Homework 9}
\begin{document}
\maketitle
\subsection*{Solution 1.9.1.}
\begin{enumerate}
    \item Proposition 10.5.16. yields $Au \otimes Bv = (A \otimes B)(u \otimes v)$. \\
    $u \otimes v$ is Kronecker product, and by proposition 10.5.14, Kronecker product is bilinear. \\
    Since $A\otimes B$ is linear, the map is the product of a linear map, and a bilinear map, which is automatically bilinear.
    \item Since it is finite dimensional, the space should be isomorphic to some vector space. \\
    Let $X$ and $Y$ in 1 to be the matrix in a vector space in which $X$ and $Y$ has Jordan blocks on its diagonal and elsewhere zero.\\
    Then $X\otimes Y$ is an upper diagonal matrix, in which the diagonal elements are $\lambda_i\mu_j$, $i = 1, 2, \cdots, \mathrm{Dim}(V), j = 1,2, \cdots, \mathrm{Dim}(W)$, in which $\lambda, \mu$ are eigenvalues of $X$ and $T$. \\
    So 
    \[\mathrm{trace}(X\otimes Y) = \sum_{i=1} ^ {\mathrm{Dim}(V)} \sum_{j = 1} ^ {\mathrm{Dim}(W)}\lambda_i\mu_j = \sum_{i=1} ^ {\mathrm{Dim}(V)} \lambda_i \sum_{j = 1} ^ {\mathrm{Dim}(W)}\mu_j=\sum_{i=1} ^ {\mathrm{Dim}(V)} \lambda_i \mathrm{trace}(Y) = \mathrm{trace}(X)\mathrm{trace}(Y)\]
\end{enumerate}
\subsection*{Solution 1.9.2.}
\[\mathrm{trace}\in \mathcal{L}(R^n \otimes (R^n)^*,R) = (R^n)^* \otimes (R ^n) ^{**} \otimes R = (R^n)^* \otimes R ^n\]
Let $r$ be the rank of a particular matrix $A$, then
\[\mathrm{trace}(A) = \sum_{i = 1} ^ r t_i \mathrm{trace} (u_iv_i^T) = \sum_{i=1}^r\sum_{j = 1}^n t_i u_i^jv_i^j\]
Therefore, it is just a dot product of the rightmost vectors.\\
Dot product is a special kind of inner product, in which it is the case for solution 1.7.1. when $A=I$, and hence
\[\mathrm{trace}_{ij} = \delta_{ij}\]
\subsection*{Solution 1.9.3.}
\begin{enumerate}
    \item
    Symmetry:
    \[(v_1\otimes w_1,v_2\otimes w_2) = (v_1,v_2)(w_1,w_2) = (v_2,v_1)(w_2,w_1) = (v_2\otimes w_2, v_1\otimes w_1)\]
    Positive definiteness:
    \[(v\otimes w, v\otimes w) = (v,v)(w,w)\]
    It is zero iff $v=0$ or $w=0$, iff $v\otimes w = 0$
    \item
    \[(e_1\otimes e_1, e_2 \otimes e_2) = (e_1,e_2)(e_1, e_2)\]
    Since $e_1, e_2$ are two basis vectors, their inner product must not be $\pm$1. (or else they lie on the same line and cannot be basis vectors).
    There are two components, so the rank must not be greater than 2. These two simple tensors are not proportional, so rank must not be less than 2.
    \item $\omega = v \otimes w$
    \[(\omega,L\otimes I_B(\omega)) = (v\otimes w, Lv \otimes w) = (v_1^2-v_2^2)(w_1^2+w_2^2)\]
    \[(\omega, I_A\otimes L(\omega)) = (v\otimes w, v \otimes Lw) = (v_1^2+v_2^2)(w_1^2-w_2^2)\]
    Let the former to be $x$, and the latter to be $y$. Partition $R^2$ into four regions by the lines $x+y = 0$ and $x - y = 0$.\\
    Case 1: $x + y \leq 0 \land x - y \leq 0$. Then let
    \[v_1=0, v_2=\pm 1 \Rightarrow w_1 = \pm\sqrt{-\frac{x-y}{2}}, w_2 = \pm\sqrt{-\frac{x+y}{2}}\]
    Case 2: $x +y \geq 0 \land x - y \geq 0$. Then let
    \[v_1 = \pm 1, v_2 = 0 \Rightarrow w_1 = \pm \sqrt{\frac{x+y}{2}}, w_2 = \pm \sqrt{\frac{x-y}{2}}\]
    Case 3: $x +y \geq 0 \land x - y \leq 0$. Then let
    \[w_1 = \pm 1, w_2 = 0 \Rightarrow v_1 = \pm \sqrt{\frac{x+y}{2}}, v_2 = \pm \sqrt{-\frac{x-y}{2}}\]
    Case 3: $x + y \leq 0 \land x - y \geq 0$. Then let
    \[w_1=0, w_2= \pm 1 \Rightarrow w_1 = \pm\sqrt{-\frac{x+y}{2}}, w_2 = \pm\sqrt{\frac{x-y}{2}}\]
    Hence it is possible to take any point of $(x,y)\in R^2$.
    \item
    \[L\otimes I_B(\omega) = L\otimes I_B(ae_1\otimes e_1) +L\otimes I_B(be_2\otimes e_2) = aLe_1\otimes e_1 + bLe_2 \otimes e_2 = ae_1\otimes e_1 - be_2\otimes e_2 \]
    \[I_A\otimes L(\omega) = I_A\otimes L(ae_1\otimes e_1) +I_A\otimes L(be_2\otimes e_2) = ae_1\otimes Le_1 + be_2 \otimes Le_2 = ae_1\otimes e_1 - be_2\otimes e_2\]
    So they are indeed identical.
\end{enumerate}
\end{document}